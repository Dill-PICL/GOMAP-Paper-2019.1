%%%%%%%%%%%%%%%%%%%%%%%%%%%%%%%%%%%%%%%%%%%%%%%%%%%%%%%%%%%%%%%%%%%%%%%%%%%%%%%%%%%%%%%%%%%%%%%%%%%%%%%%%%%%%%%%%%%%%%%%%%%%%%%%%%%%%%%%%%%%%%%%%%%%%%%%%%%
% This is just an example/guide for you to refer to when submitting manuscripts to Frontiers, it is not mandatory to use Frontiers .cls files nor frontiers.tex  %
% This will only generate the Manuscript, the final article will be typeset by Frontiers after acceptance.   
%                                              %
%                                                                                                                                                         %
% When submitting your files, remember to upload this *tex file, the pdf generated with it, the *bib file (if bibliography is not within the *tex) and all the figures.
%%%%%%%%%%%%%%%%%%%%%%%%%%%%%%%%%%%%%%%%%%%%%%%%%%%%%%%%%%%%%%%%%%%%%%%%%%%%%%%%%%%%%%%%%%%%%%%%%%%%%%%%%%%%%%%%%%%%%%%%%%%%%%%%%%%%%%%%%%%%%%%%%%%%%%%%%%%

%%% Version 3.4 Generated 2018/06/15 %%%
%%% You will need to have the following packages installed: datetime, fmtcount, etoolbox, fcprefix, which are normally inlcuded in WinEdt. %%%
%%% In http://www.ctan.org/ you can find the packages and how to install them, if necessary. %%%
%%%  NB logo1.jpg is required in the path in order to correctly compile front page header %%%

\documentclass[utf8,]{frontiersSCNS} % for Science, Engineering and Humanities and Social Sciences articles
%\documentclass[utf8]{frontiersHLTH} % for Health articles
%\documentclass[utf8]{frontiersFPHY} % for Physics and Applied Mathematics and Statistics articles

%\setcitestyle{square} % for Physics and Applied Mathematics and Statistics articles
\usepackage{url,hyperref,lineno,microtype,subcaption}
\usepackage[onehalfspacing]{setspace}


\linenumbers


% Leave a blank line between paragraphs instead of using \\


\def\keyFont{\fontsize{8}{11}\helveticabold }
\def\firstAuthorLast{Sample {et~al.}} %use et al only if is more than 1 author
\def\Authors{First Author}
% Affiliations should be keyed to the author's name with superscript numbers and be listed as follows: Laboratory, Institute, Department, Organization, City, State abbreviation (USA, Canada, Australia), and Country (without detailed address information such as city zip codes or street names).
% If one of the authors has a change of address, list the new address below the correspondence details using a superscript symbol and use the same symbol to indicate the author in the author list.
\def\Address{Address}
% The Corresponding Author should be marked with an asterisk
% Provide the exact contact address (this time including street name and city zip code) and email of the corresponding author
\def\corrAuthor{Corresponding Author}

\def\corrEmail{email@uni.edu}


\usepackage{booktabs}
\usepackage{longtable}
\usepackage{array}
\usepackage{multirow}
\usepackage{wrapfig}
\usepackage{float}
\usepackage{colortbl}
\usepackage{pdflscape}
\usepackage{tabu}
\usepackage{threeparttable}
\usepackage{threeparttablex}
\usepackage[normalem]{ulem}
\usepackage{makecell}
\usepackage{xcolor}


\begin{document}
\onecolumn
\firstpage{1}

\title[Running Title]{TBD} 

\author[\firstAuthorLast ]{\Authors} %This field will be automatically populated
\address{} %This field will be automatically populated
\correspondance{} %This field will be automatically populated

\extraAuth{}% If there are more than 1 corresponding author, comment this line and uncomment the next one.
%\extraAuth{corresponding Author2 \\ Laboratory X2, Institute X2, Department X2, Organization X2, Street X2, City X2 , State XX2 (only USA, Canada and Australia), Zip Code2, X2 Country X2, email2@uni2.edu}


\maketitle


\begin{abstract}

Hello, this is the abstract.

  \tiny
   \keyFont{ \section{Keywords:} keyword1, kw2, kw3\ldots{} have to have at least 5, max 8} %All article types: you may provide up to 8 keywords; at least 5 are mandatory.

\end{abstract}

\section{Results}\label{results}

\ldots{} a quantitative comparison of the datasets in Table
\ref{tab:t_annotation_quantities}.

\begin{table}[t]

\caption{\label{tab:t_annotation_quantities}Quantitative metrics of the generated and existing functional annotation sets. C, F, P, and A refer to the aspects of the GO: Cellular Component, Biological Function, Molecular Process, and Any/All.}
\centering
\resizebox{\linewidth}{!}{
\begin{threeparttable}
\begin{tabular}{lrlrrr>{\bfseries}r|rrr>{\bfseries}r|rrr>{\bfseries}r}
\toprule
\multicolumn{3}{c}{ } & \multicolumn{4}{c}{Annotations\textsuperscript{a}} & \multicolumn{4}{c}{Annotated Genes\textsuperscript{b}} & \multicolumn{4}{c}{Median Ann. per G.\textsuperscript{c}} \\
\cmidrule(l{3pt}r{3pt}){4-7} \cmidrule(l{3pt}r{3pt}){8-11} \cmidrule(l{3pt}r{3pt}){12-15}
Genome & Genes & Dataset & C & F & P & A & C & F & P & A & C & F & P & A\\
\midrule
\rowcolor{gray!6}   &  & GOMAP & 135251 & 87953 & 291855 & 515059 & 34867 & 38099 & 39469 & 39469 & 3 & 2 & 6 & 11\\

 &  & Gramene49 & 20072 & 31139 & 30102 & 81313 & 11834 & 18033 & 15800 & 21926 & 1 & 1 & 1 & 3\\

\rowcolor{gray!6}  \multirow{-3}{*}{\raggedright\arraybackslash maize\_v3} & \multirow{-3}{*}{\raggedleft\arraybackslash 300} & Phytozome & 4787 & 19098 & 13100 & 36985 & 4524 & 13775 & 11365 & 16132 & 0 & 1 & 1 & 2\\
\cmidrule{1-15}
maize\_v4 & 320 & GOMAP & 88831 & 82849 & 278952 & 450632 & 36717 & 37431 & 39324 & 39324 & 2 & 2 & 6 & 10\\
\cmidrule{1-15}
\rowcolor{gray!6}  rice & 200 & GOMAP & 72782 & 64783 & 248713 & 386278 & 28619 & 29876 & 35824 & 35825 & 2 & 2 & 6 & 9\\
\cmidrule{1-15}
wheat & 500 & GOMAP & 267742 & 218839 & 786028 & 1272609 & 95604 & 98224 & 107890 & 107891 & 2 & 2 & 6 & 10\\
\bottomrule
\end{tabular}
\begin{tablenotes}
\item[a] How many annotations in the C, F, and P aspect does this dataset contain? A = How many in total? $A = C + F + P$
\item[b] How many genes in the genome have at least one GO term from the C, F, P aspect annotated to them? A = How many at least one from any aspect? ($A = C \cup F \cup P$)
\item[c] Take a typical gene that is present in the annotation set. How many annotations does it have in each aspect? A = How many in total? Ask your favorite statistician why $A \neq C + F +P$
\end{tablenotes}
\end{threeparttable}}
\end{table}

\end{document}
