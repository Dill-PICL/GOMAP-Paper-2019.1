\documentclass[]{article}
\usepackage{lmodern}
\usepackage{amssymb,amsmath}
\usepackage{ifxetex,ifluatex}
\usepackage{fixltx2e} % provides \textsubscript
\ifnum 0\ifxetex 1\fi\ifluatex 1\fi=0 % if pdftex
  \usepackage[T1]{fontenc}
  \usepackage[utf8]{inputenc}
\else % if luatex or xelatex
  \ifxetex
    \usepackage{mathspec}
  \else
    \usepackage{fontspec}
  \fi
  \defaultfontfeatures{Ligatures=TeX,Scale=MatchLowercase}
\fi
% use upquote if available, for straight quotes in verbatim environments
\IfFileExists{upquote.sty}{\usepackage{upquote}}{}
% use microtype if available
\IfFileExists{microtype.sty}{%
\usepackage{microtype}
\UseMicrotypeSet[protrusion]{basicmath} % disable protrusion for tt fonts
}{}
\usepackage[margin=1in]{geometry}
\usepackage{hyperref}
\hypersetup{unicode=true,
            pdftitle={TBD},
            pdfauthor={The Pickles},
            pdfborder={0 0 0},
            breaklinks=true}
\urlstyle{same}  % don't use monospace font for urls
\usepackage{graphicx,grffile}
\makeatletter
\def\maxwidth{\ifdim\Gin@nat@width>\linewidth\linewidth\else\Gin@nat@width\fi}
\def\maxheight{\ifdim\Gin@nat@height>\textheight\textheight\else\Gin@nat@height\fi}
\makeatother
% Scale images if necessary, so that they will not overflow the page
% margins by default, and it is still possible to overwrite the defaults
% using explicit options in \includegraphics[width, height, ...]{}
\setkeys{Gin}{width=\maxwidth,height=\maxheight,keepaspectratio}
\IfFileExists{parskip.sty}{%
\usepackage{parskip}
}{% else
\setlength{\parindent}{0pt}
\setlength{\parskip}{6pt plus 2pt minus 1pt}
}
\setlength{\emergencystretch}{3em}  % prevent overfull lines
\providecommand{\tightlist}{%
  \setlength{\itemsep}{0pt}\setlength{\parskip}{0pt}}
\setcounter{secnumdepth}{0}
% Redefines (sub)paragraphs to behave more like sections
\ifx\paragraph\undefined\else
\let\oldparagraph\paragraph
\renewcommand{\paragraph}[1]{\oldparagraph{#1}\mbox{}}
\fi
\ifx\subparagraph\undefined\else
\let\oldsubparagraph\subparagraph
\renewcommand{\subparagraph}[1]{\oldsubparagraph{#1}\mbox{}}
\fi

%%% Use protect on footnotes to avoid problems with footnotes in titles
\let\rmarkdownfootnote\footnote%
\def\footnote{\protect\rmarkdownfootnote}

%%% Change title format to be more compact
\usepackage{titling}

% Create subtitle command for use in maketitle
\providecommand{\subtitle}[1]{
  \posttitle{
    \begin{center}\large#1\end{center}
    }
}

\setlength{\droptitle}{-2em}

  \title{TBD}
    \pretitle{\vspace{\droptitle}\centering\huge}
  \posttitle{\par}
    \author{The Pickles}
    \preauthor{\centering\large\emph}
  \postauthor{\par}
      \predate{\centering\large\emph}
  \postdate{\par}
    \date{Febcember 32nd 3023}

\usepackage{booktabs}
\usepackage{longtable}
\usepackage{array}
\usepackage{multirow}
\usepackage{wrapfig}
\usepackage{float}
\usepackage{colortbl}
\usepackage{pdflscape}
\usepackage{tabu}
\usepackage{threeparttable}
\usepackage{threeparttablex}
\usepackage[normalem]{ulem}
\usepackage{makecell}
\usepackage{xcolor}

\begin{document}
\maketitle

\hypertarget{methods}{%
\section{Methods}\label{methods}}

\hypertarget{clean-up}{%
\subsection{Clean up}\label{clean-up}}

All functional annotation sets were cleaned up the following way (using
definitions from the Gene Ontology version 2019-07-01):

\begin{enumerate}
\def\labelenumi{\arabic{enumi}.}
\tightlist
\item
  Any annotations where the GO accession was marked as obsolete were
  removed.
\item
  Some terms in the GO have `alternative ids'. When naively removing
  duplicates, two entries will not be recognized as duplicates if they
  have different accessions pointing to the same GO term. Therefore, all
  GO accessions were changed to their respecitve `main id' and the
  dataset was again scanned for duplicates.
\end{enumerate}

Table 1 provides information on the number of annotations that were
removed this way from each dataset. All further analyses were performed
on the cleaned datasets since we assume the user will only be interested
in still valid and non-redundant functional annotations.

\hypertarget{results}{%
\section{Results}\label{results}}

\ldots{} a quantitative comparison of the datasets in Table.

\begin{table}[t]

\caption{\label{tab:cleanup_table}Number of removed annotations during cleanup.}
\centering
\begin{tabular}{llrr}
\toprule
Genome & Dataset & Obsolete Annotations & Duplicates\\
\midrule
\rowcolor{gray!6}  Triticum\_aestivum & GOMAP & 285 & 0\\
\cmidrule{1-4}
Hordeum\_vulgarum & GOMAP & 101 & 0\\
\cmidrule{1-4}
\rowcolor{gray!6}  Glycine\_max & GOMAP & 203 & 0\\
\cmidrule{1-4}
Arachis\_hypogaea & GOMAP & 0 & 0\\
\cmidrule{1-4}
\rowcolor{gray!6}  Zea\_mays.PH207 & GOMAP & 798 & 76\\
\cmidrule{1-4}
Zea\_mays.Mo17 & GOMAP & 726 & 77\\
\cmidrule{1-4}
\rowcolor{gray!6}  Phaseolus\_vulgaris & GOMAP & 0 & 0\\
\cmidrule{1-4}
Vigna\_unguiculata & GOMAP & 0 & 0\\
\cmidrule{1-4}
\rowcolor{gray!6}  Medicago\_truncatula.R108 & GOMAP & 0 & 0\\
\cmidrule{1-4}
Zea\_mays.B73.v3 & GOMAP & 1107 & 70\\
\cmidrule{1-4}
\rowcolor{gray!6}  Zea\_mays.W22 & GOMAP & 754 & 82\\
\cmidrule{1-4}
Medicago\_truncatula.A17 & GOMAP & 0 & 0\\
\cmidrule{1-4}
\rowcolor{gray!6}  Zea\_mays.B73.v4 & GOMAP & 752 & 83\\
\cmidrule{1-4}
 & GoldStandard & 38 & 556\\

\rowcolor{gray!6}  \multirow{-2}{*}{\raggedright\arraybackslash Oryza\_sativa} & GOMAP & 111 & 2\\
\bottomrule
\end{tabular}
\end{table}

\begin{table}[t]

\caption{\label{tab:t_annotation_quantities}Quantitative metrics of the cleaned functional annotation sets. C, F, P, and A refer to the aspects of the GO: Cellular Component, Biological Function, Molecular Process, and Any/All.}
\centering
\resizebox{\linewidth}{!}{
\begin{threeparttable}
\begin{tabular}{lrlrrr>{\bfseries}r|rrr>{\bfseries}r|rrr>{\bfseries}r}
\toprule
\multicolumn{3}{c}{ } & \multicolumn{4}{c}{Annotations\textsuperscript{a}} & \multicolumn{4}{c}{Annotated Genes [\%]\textsuperscript{b}} & \multicolumn{4}{c}{Median Ann. per G.\textsuperscript{c}} \\
\cmidrule(l{3pt}r{3pt}){4-7} \cmidrule(l{3pt}r{3pt}){8-11} \cmidrule(l{3pt}r{3pt}){12-15}
Genome & Genes & Dataset & C & F & P & A & C & F & P & A & C & F & P & A\\
\midrule
\rowcolor{gray!6}  Arachis\_hypogaea &  & GOMAP & 153433 & 132944 & 493799 & 780176 & 576.67 & 568.55 & 671.23 & 671.24 & 2 & 2 & 6 & 10\\

Glycine\_max &  & GOMAP & 129215 & 113827 & 417555 & 660597 & 460.20 & 470.34 & 528.71 & 528.72 & 2 & 2 & 6 & 11\\

\rowcolor{gray!6}  Hordeum\_vulgarum &  & GOMAP & 88130 & 80282 & 272823 & 441235 & 352.37 & 364.70 & 397.33 & 397.34 & 2 & 2 & 5 & 10\\

Medicago\_truncatula.A17 &  & GOMAP & 107362 & 99719 & 364065 & 571146 & 423.25 & 437.36 & 504.43 & 504.44 & 2 & 2 & 6 & 10\\

\rowcolor{gray!6}  Medicago\_truncatula.R108 &  & GOMAP & 112343 & 108031 & 382322 & 602696 & 403.32 & 502.20 & 557.06 & 557.06 & 1 & 2 & 5 & 9\\

 &  & GOMAP & 72780 & 64685 & 248700 & 386165 & 286.19 & 298.53 & 358.24 & 358.25 & 2 & 2 & 6 & 9\\

\rowcolor{gray!6}  \multirow{-2}{*}{\raggedright\arraybackslash Oryza\_sativa} &  & GoldStandard & 7730 & 11060 & 19378 & 38176 & 57.25 & 73.83 & 90.31 & 113.87 & 1 & 1 & 1 & 3\\

Phaseolus\_vulgaris &  & GOMAP & 72005 & 64583 & 229630 & 366218 & 259.34 & 255.39 & 274.32 & 274.33 & 2 & 2 & 6 & 11\\

\rowcolor{gray!6}  Triticum\_aestivum &  & GOMAP & 267741 & 218623 & 785960 & 1272324 & 956.04 & 981.87 & 1078.90 & 1078.91 & 2 & 2 & 6 & 10\\

Vigna\_unguiculata &  & GOMAP & 75867 & 68313 & 243278 & 387458 & 271.73 & 271.24 & 297.72 & 297.73 & 2 & 2 & 6 & 11\\

\rowcolor{gray!6}  Zea\_mays.B73.v3 &  & GOMAP & 135211 & 87420 & 291251 & 513882 & 348.66 & 380.73 & 394.68 & 394.69 & 3 & 2 & 6 & 11\\

Zea\_mays.B73.v4 &  & GOMAP & 88827 & 82251 & 278719 & 449797 & 367.17 & 373.37 & 393.23 & 393.24 & 2 & 2 & 6 & 10\\

\rowcolor{gray!6}  Zea\_mays.Mo17 &  & GOMAP & 87567 & 79214 & 277787 & 444568 & 336.18 & 351.05 & 386.19 & 386.20 & 2 & 2 & 6 & 10\\

Zea\_mays.PH207 &  & GOMAP & 90617 & 85500 & 288677 & 464794 & 351.70 & 367.62 & 405.56 & 405.57 & 2 & 2 & 6 & 10\\

\rowcolor{gray!6}  Zea\_mays.W22 & \multirow{-15}{*}{\raggedleft\arraybackslash 100} & GOMAP & 95390 & 85039 & 289780 & 470209 & 369.87 & 376.85 & 406.89 & 406.90 & 2 & 2 & 6 & 10\\
\bottomrule
\end{tabular}
\begin{tablenotes}
\item[a] How many annotations in the C, F, and P aspect does this dataset contain? A = How many in total? $A = C + F + P$
\item[b] How many genes in the genome have at least one GO term from the C, F, P aspect annotated to them? A = How many at least one from any aspect? ($A = C \cup F \cup P$)
\item[c] Take a typical gene that is present in the annotation set. How many annotations does it have in each aspect? A = How many in total? Ask your favorite statistician why $A \neq C + F +P$
\end{tablenotes}
\end{threeparttable}}
\end{table}


\end{document}
